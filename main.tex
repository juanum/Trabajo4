\documentclass[a4paper]{article}

\usepackage[english]{babel}
\usepackage[utf8]{inputenc}
\usepackage{amsmath}
\usepackage{graphicx}
\usepackage[colorinlistoftodos]{todonotes}
\usepackage{titling}

\title{Trabajo 4. Documento Latex usando Matlab.}
\author{Alumno: Juan Antonio Rodríguez Lorente DNI: 48854839X}
\date{}

\begin{document}

\maketitle


\section{Introduccion}
El problema a abordar va a ser la resolucion en matlab de la ecuacion:
\begin{equation}
f(x)=x+log(x)
\end{equation}
Es decir, queremos encontrar el punto $\overline{x}$ tal que f($\overline{x}$)=0.

\section{Funciones usadas}

\section{Resolucion del Problema}
Primero definimos tres metodos iterativos:
\begin{equation}
\begin{array}{ccc}
\mathrm{I1:}x_{n+1}=log(x_{n}) & \mathrm{I2:}x_{n+1}=e^{-x_{n}} & \mathrm{I3:}x_{n+1}=\frac{x_{n}+e^{-x_{n}}}{2}
\end{array}
\end{equation}
Que surgen de despejar x de distintas formas. Ahora estas 3 iteraciones son iteraciones de punto fijo. Con lo cual primero tenemos que ver cuan de los tres es el mas optimo.

Primero vamos a ver donde estará aproximadamente la raíz de la ecuación. Como $f(0)<0$ y $f(1)>0$, entonces por Bolzano la raíz estará en el intervalo [0,1].

Sungundo, definimos como:
\begin{equation}
\begin{array}{ccc}
g_{1}:=log(x) & g_{2}:=e^{x} & g_{3}:=\frac{x+e^{-x}}{2}
\end{array}
\end{equation}
Ahora hay que ver cual de ellas tiene derivada menor que 1 para que el metodo del punto fijo funcione. Derivando cada uno tenemos que:
\begin{equation}
\begin{array}{ccc}
max\{|g_{1}'|\}_{[0,1]}=+\infty & max\{|g_{2}'|\}_{[0,1]}=0.36 & max\{|g_{3}'|\}_{[0,1]}=0.31
\end{array}
\end{equation}
Vemos que las funciones que cumplen la condición son $g_{3}$ y $g_{2}$ y como $g_{3}$ tiene menor valor,el más óptimo es $g_{3}$.

Por ultimo solo tenemos que poner en matlab que ejecute el metodo del punto fijo en el punto inicial 0 y nos da como resultado: 0.56714329.


\end{document}
